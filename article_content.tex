\section{Introduction}
A single paragraph of meaningless text.  \lipsum[1]


This is the introduction. Some Latex-specific formatting tips are provided here. Proper quotation marks are done ``this way''. 

A line break requires an empty line between two lines of text in your code. Do not use \verb+\newline+ to force a new paragraph.

There is a difference between three types of dashes: (1) simple dash for composite words such as Latex-specific; (2) double dash for numerical range such as 204--2014; (3) long dashes---I am sure you have seen them before---for included sentences.

Bullet point environments:
\begin{itemize}
\item In New York
\item Rio
\item Tokyo
\end{itemize}

Enumerated environments:
\begin{enumerate}
\item Most important
\item Less important
\item Not important
\end{enumerate}

\subsection{Citing literature}
We can refer to papers directly. I greatly admire the study by \cite{ADRIAN1934}. We can also cite papers in parenthesis. Other studies are great, too \citep{Albrecht1982}. Sometimes, we want to write something additionally into the parentheses \citep[the study by][is also not bad]{Allard2011}.

\subsection{Within-paper references}
\label{withpaperrefs}
You can easily refer to sections within the paper. This is explained in section \ref{withpaperrefs}. You can also refer to figures. The numbering is done fully automatically, see Figure \ref{samplefig}.


\begin{figure}[t]
\centering
\includegraphics[width=1\textwidth]{Figures/resultspdf}
\caption{This is the figure caption. Note that figures in vector format (.eps; .pdf) are preferred over bitmap formats (.jpg).}
\label{samplefig}
\end{figure}